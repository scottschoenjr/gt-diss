\chapter{Introduction and Background}

\section{About this Template}
This template is an adaptation \texttt{gatechthesis\_latex} template (17 January 2017 update) and \texttt{ut-diss-2}. 
I've made some (mostly) aesthetic changes, but I believe it should conform to the requirements liseted in the Georgia Tech ``Graduate Thesis/Dissertation Guidelines \& Procedures'' (April 2015 Update).
It is by no means an official template, though. 
So as stated in the license file, you can use, change, distribute, sell it however you'd like, provided
\begin{enumerate}
  \item You include the copyright notice in the license file; and
  \item You hold none of the authors liable for any results of such use.\cite{ref2}
\end{enumerate}

\section{General Philosophy}
One of the big advantages of \LaTeX{} is the ability to automate anything and everything about the formatting document. 
For theses this can be great, as getting the titles, headings, table of contents, etc. to conform to the formatting requirements.
However, this sometimes has the drawback of making alternations a pain. \\

This template has been organized to strike as happy a medium as I can see. 
Lots of the stuff is automated so that if you're happy with the output, you never have to think about it. 
But, if you want to customize and tweak (as I've done with the baseline templates mentioned above) it should be a least reasonably straightforward.
Sections (e.g., the abstract, epigraph, etc.) were given their own files, and the preamble and main .tex files are organized to give easy access to the settings. 
All of this is to say, I've avoided using class and style files, as in my experience they make small adjustments a big effort.
Whether this is an optimal choice I won't argue, but I like it, so don't $@$me
